\documentclass[25pt, a0paper, portrait, margin=0mm, innermargin=15mm, blockverticalspace=15mm, colspace=15mm, subcolspace=8mm]{tikzposter}

\usepackage[utf8]{inputenc}
\usepackage{amsmath}
\usepackage{amsfonts}
\usepackage{amsthm}
\usepackage{amssymb}
\usepackage{mathrsfs}
\usepackage{graphicx}
\usepackage{adjustbox}
\usepackage{enumitem}
\usepackage{tikz}
\usepackage{booktabs}
\usepackage{array}
\usepackage{xcolor}

% Define custom colors
\definecolor{arithmaxblue}{RGB}{0,102,204}
\definecolor{arithmaxgray}{RGB}{128,128,128}
\definecolor{performancegreen}{RGB}{0,128,0}
\definecolor{warningred}{RGB}{204,0,0}

% Set the theme
\usetheme{Simple}
\usecolorstyle[colorPalette=GrayOrangePalette]{Germany}

% Override some colors
\colorlet{backgroundcolor}{white}
\colorlet{framecolor}{arithmaxblue}
\colorlet{titlefgcolor}{white}
\colorlet{titlebgcolor}{arithmaxblue}
\colorlet{blocktitlebgcolor}{arithmaxblue}
\colorlet{blocktitlefgcolor}{white}

\title{\parbox{\linewidth}{\centering Algorithmic Rebalancing for Multi-Asset Leveraged Index Funds}}
\author{Frankline Misango Oyolo \\ Arithmax Research}
\date{July 2025}
\institute{research@arithmax.com}

\begin{document}

\maketitle

\begin{columns}
    \column{0.5}
    
    \block{Executive Summary}{
        \textbf{Strategy Overview:} A diversified leverage framework that captures enhanced returns through leveraged ETFs while managing downside risk via strategic asset allocation.
        
        \vspace{10pt}
        \textbf{Key Innovation:}
        \begin{itemize}[leftmargin=15pt]
            \item Dynamic rebalancing with momentum preservation
            \item Multi-asset risk parity across leveraged instruments  
            \item Volatility-adaptive allocation methodology
            \item Tariff-resilient sector diversification
        \end{itemize}
        
        \vspace{10pt}
        \textbf{Performance Highlights (2020-2024):}
        \begin{itemize}[leftmargin=15pt]
            \item \textcolor{performancegreen}{\textbf{686.58\% Total Return}} (51.0\% Annual)
            \item \textcolor{performancegreen}{\textbf{Sharpe Ratio: 0.701}} (vs 0.65 S\&P 500)
            \item \textcolor{performancegreen}{\textbf{Win Rate: 62\%}} with 2.81 P/L ratio
            \item Maximum Drawdown: -66.4\%
        \end{itemize}
    }
    
    \block{Mathematical Framework}{
        \textbf{Portfolio Optimization:}
        \begin{equation*}
            \mathbf{w}_t = \argmin_{\mathbf{w}} \left[ \mathbf{w}^T \boldsymbol{\Sigma}_t \mathbf{w} - \lambda \boldsymbol{\mu}_t^T \mathbf{w} \right]
        \end{equation*}
        
        \textbf{Risk Parity Allocation:}
        \begin{equation*}
            w_i = \frac{1/\sigma_i}{\sum_{j=1}^n 1/\sigma_j} \cdot \frac{1}{\beta_i}
        \end{equation*}
        
        \textbf{Leveraged ETF Dynamics:}
        \begin{equation*}
            R_{ETF,t} = \beta \cdot R_{index,t} - C_t
        \end{equation*}
        
        \textbf{Volatility Decay Control:}
        \begin{equation*}
            \mathbb{E}[\text{Decay}] = \frac{\beta(\beta-1)}{2} \sigma^2_{index} \Delta t
        \end{equation*}
    }
    
    \block{Asset Allocation Strategy}{
        \begin{center}
        \begin{tabular}{lccc}
        \toprule
        \textbf{Symbol} & \textbf{Asset Class} & \textbf{Leverage} & \textbf{Weight} \\
        \midrule
        TQQQ & Technology (Nasdaq) & 3x & 20\% \\
        UPRO & Large Cap (S\&P 500) & 3x & 20\% \\
        UDOW & Blue Chip (Dow Jones) & 3x & 10\% \\
        TMF & Treasury Bonds & 3x & 25\% \\
        UGL & Gold & 2x & 10\% \\
        DIG & Energy Sector & 2x & 15\% \\
        \bottomrule
        \end{tabular}
        \end{center}
        
        \vspace{15pt}
        \textbf{Strategic Rationale:}
        \begin{itemize}[leftmargin=15pt]
            \item \textbf{Technology (50\%):} Growth resilience, innovation premium
            \item \textbf{Bonds (25\%):} Defensive positioning, rate sensitivity hedge
            \item \textbf{Energy (15\%):} Inflation protection, tariff resilience
            \item \textbf{Gold (10\%):} Currency debasement hedge, portfolio stabilizer
        \end{itemize}
    }
    
    \column{0.5}
    
    \block{Performance Analysis}{
        \textbf{Risk-Adjusted Returns vs Benchmarks:}
        
        \vspace{10pt}
        \begin{center}
        \begin{tabular}{lccc}
        \toprule
        \textbf{Strategy} & \textbf{Annual Ret.} & \textbf{Sharpe} & \textbf{Max DD} \\
        \midrule
        \textcolor{performancegreen}{\textbf{Diversified Leverage}} & \textcolor{performancegreen}{\textbf{51.0\%}} & \textcolor{performancegreen}{\textbf{0.701}} & \textcolor{warningred}{\textbf{-66.4\%}} \\
        60/40 Portfolio & 8.2\% & 0.68 & -8.9\% \\
        S\&P 500 & 10.5\% & 0.65 & -19.6\% \\
        TQQQ Only & 28.7\% & 0.81 & -28.1\% \\
        \bottomrule
        \end{tabular}
        \end{center}
        
        \vspace{15pt}
        \textbf{Key Performance Metrics:}
        \begin{itemize}[leftmargin=15pt]
            \item \textbf{Sortino Ratio:} 1.998 (excellent downside control)
            \item \textbf{Alpha:} 0.683 (significant market outperformance)
            \item \textbf{Beta:} 1.223 (moderate systematic risk)
            \item \textbf{Total Trades:} 2,289 over 5 years
            \item \textbf{Transaction Costs:} Only \$2,427 total
        \end{itemize}
        
        \vspace{15pt}
        \textbf{Market Regime Performance:}
        \begin{center}
        \begin{tabular}{lcc}
        \toprule
        \textbf{Market Condition} & \textbf{Expected Return} & \textbf{Volatility} \\
        \midrule
        Bull Market & 25.2\% & 18.5\% \\
        Bear Market & -8.3\% & 32.1\% \\
        High Volatility & 12.1\% & 28.7\% \\
        Tariff Uncertainty & 15.8\% & 22.4\% \\
        \bottomrule
        \end{tabular}
        \end{center}
    }
    
    \block{Risk Management Framework}{
        \textbf{Dynamic Risk Controls:}
        \begin{itemize}[leftmargin=15pt]
            \item \textbf{Position Limits:} Maximum 30\% single asset concentration
            \item \textbf{Volatility Adjustment:} Reduce positions when vol > 1.5x historical
            \item \textbf{Drawdown Protection:} Defensive mode at 10\% portfolio decline
            \item \textbf{Correlation Monitoring:} Rebalance when correlations > 0.8
        \end{itemize}
        
        \vspace{15pt}
        \textbf{Rebalancing Algorithm:}
        \begin{itemize}[leftmargin=15pt]
            \item \textbf{Frequency:} Every 4 trading days
            \item \textbf{Threshold:} Minimum trade size for cost efficiency
            \item \textbf{Timing:} Market open execution to minimize impact
            \item \textbf{Cost Control:} Total implementation cost < 0.10\% annually
        \end{itemize}
        
        \vspace{15pt}
        \textbf{Stress Testing Results:}
        \begin{center}
        \begin{tabular}{lc}
        \toprule
        \textbf{Scenario} & \textbf{Portfolio Impact} \\
        \midrule
        Market Crash (-30\%) & -18.2\% \\
        Rate Shock (+200bp) & -15.4\% \\
        Tariff Escalation (+25\%) & +8.3\% \\
        Liquidity Crisis (-50\%) & 2.8\% TE \\
        \bottomrule
        \end{tabular}
        \end{center}
    }
    
    \block{Key Insights \& Future Research}{
        \textbf{Strategic Advantages:}
        \begin{itemize}[leftmargin=15pt]
            \item Enhanced returns through intelligent leverage application
            \item Risk mitigation via multi-asset diversification  
            \item Systematic momentum capture through rebalancing
            \item Adaptive allocation for changing market conditions
        \end{itemize}
        
        \vspace{15pt}
        \textbf{Research Contributions:}
        \begin{itemize}[leftmargin=15pt]
            \item Novel risk parity framework for leveraged instruments
            \item Quantitative approach to tariff-resilient positioning
            \item Optimal rebalancing frequency under transaction costs
            \item Volatility decay mitigation through diversification
        \end{itemize}
        
        \vspace{15pt}
        \textbf{Future Enhancements:}
        \begin{itemize}[leftmargin=15pt]
            \item Machine learning integration for regime detection
            \item LSTM networks for dynamic correlation forecasting
            \item Alternative data incorporation (sentiment analysis)
            \item Behavioral finance utility function optimization
        \end{itemize}
        
        \vspace{15pt}
        \textcolor{arithmaxblue}{\textbf{Contact:}} research@arithmax.com \textbar{} \textcolor{arithmaxblue}{\textbf{LinkedIn:}} /in/frankline-misango
    }
    
\end{columns}

\end{document}
